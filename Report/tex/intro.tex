In recent years, Unmanned Aerial Vehicles (UAVs) have been the subject of many research projects and public scrutiny. While fixed wing UAVs have been successfully deployed in several different scenarios \textbf{[citation needed]}, rotor blade ones are still being actively researched and improved upon. Helicopters have long been the preferred type of rotor blade vehicle for manned transportation, mostly due to having a design that has better stability properties \textbf{[citation needed]}. 

\textcolor{red}{Develop UAV history. Explain helicopters better. Introduce other rotorblade helicopters}.

The concept of a quadcopter is not new \textbf{[citation needed]}. It was introduced in \textbf{??} and several models were built since then \textbf{[citation needed]}. It consists on a 'X' or plus shapped frame, with a rotor at each end. Each rotating blade produces air drag that counteracts its movement. To prevent the vehicle from spinning on the opposite direction of the blades direction of rotation, the rotors are assigned in pairs along the same axis of the quadcopter frame, with each pair rotating in opposite directions. The end effect is that, if the thrust produced by each rotor is the same, then the air drags created by one pair are compensated by the other's. Introducing a difference between the rotors thrust in one axis creates a torque around the other one, and creating a difference in each pair's total thrust originates a drag force that makes the frame rotate around itself. 

These easy to understand concepts makes the quadcopter an interesting platform to study, as well as its high maneuverability that makes it an ideal platform for indoors flights \textbf{[citation needed]}. However, its inherently unstable nature prevented it to be used until the recent past \textbf{[citation needed]}. Nowadays, with the availability of inexpensive embedded computers, it is possible to equip these vehicles with onboard controllers that ensure that the system will behave in a desirable manner over a pilot (or autopilot) commands. 

One major limitation of quadcopter systems relies in their autonomy \textbf{[citation needed]}. It is common to have a flight autonomy in the order of $10$ to $20$ minutes. This is due to the fact that the four rotor blades configuration is not the most power efficient one \textbf{[citation needed]}. If one wants to introduce an onboard attitude controller, the problem is magnified by the need to constantly having to compute a control signal to give to the motor controllers. One possible solution to reduce the impact of this problem is to try to minimize the number of control input updates by only computing a new control signal when some condition is fulfilled. This is called Event-Triggered control. 

In this project, a saturating controller \cite{lohmann_attitude} is implemented in a quadcopter platform, and an event triggering technique is proposed and tested.

\textcolor{red}{Develop EVT? More about the project when the project is done...}
